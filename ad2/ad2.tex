\documentclass[a4paper,12pt]{article}
\usepackage[utf8]{inputenc}
\usepackage[usenames]{xcolor}
\newcommand{\ub}[1]{\textcolor{blue}{#1}}
\newcommand{\ubc}[1]{\textcolor{brown}{#1}}
\usepackage[]

\begin{document}
\title{Übersicht Aufgaben AD2}
\author{Florian Bruhin}
\maketitle

\section{BSTs}

\begin{itemize}
  \item \ub{U1.1} Aufbau
  \item \ub{U1.2} Laufzeitverhalten Sortierung mit BST
  \item \ub{U1.3} Gültige Suchpfade
  \item \ubc{U1.4} Binary search in Array vs. Baum
  \item \ubc{U2.1} Höhe experimentell bestimmen
  \item \ub{HS05.1 5} Einfüge-Reihenfolge
  \item \ub{HS04 1} Richtig/Falsch-Fragen
\end{itemize}

\section{AVL}

\begin{itemize}
  \item \ub{PP 1} Rotation nach Löschen
  \item \ub{PP 2} Cut/Link
  \item \ub{U3.1} Balancierungsfaktor bestimmen
  \item \ub{U3.1} Insert
  \item \ub{U3.2} Cut/Link
  \item \ubc{U3.3} AVL als Map mit BST (ohne Rotationen)
  \item \ubc{U4.1} AVL-Implementation mit Rotationen
  \item \ub{HS05.1 6} Einfügen/Löschen
  \item \ub{HS04 1} Richtig/Falsch-Fragen
  \item \ub{HS04 7} AVL-Rotationen
\end{itemize}

\section{Splay}

\begin{itemize}
  % FIXME übungen?
  \item \ub{FS08 3} Laufzeitverhalten Löschen in Splay-Tree
\end{itemize}

\section{Sortierung}

\begin{itemize}
  \item \ub{PP 3} Lexikographische Sortierung
  \item \ubc{5.1} Bubble-Sort
  \item \ubc{5.2} Merge-Sort
  \item \ub{5.3} Quicksort-Analysen: Anzahl Vergleiche, Wahl des Pivots, Stabilität
  \item \ubc{6.1} Quicksort: Punkte in der Ebene, mehrfach-Keys
  \item \ubc{6.2} Radix-Sort
  \item \ub{HS05.1 3} Zeitverhalten Insertion/Selection Sort
  \item \ub{HS04 5} Insertion Sort mit Zahlensequenz
  \item \ubc{FS14 1blatt} Quicksort-Implementation
  \item \ub{HS05.2 1} Mergesort-Baum zeichnen
  \item \ub{HS05.2 2} Mergesort Höhe Beweis
  \item \ub{HS05.2 3} Quicksort
  \item \ubc{HS05.2 4} Bucket-Sort
\end{itemize}

\section{Pattern matching}

\begin{itemize}
  \item \ub{PP 4} Boyer-Moore
  \item \ub{7.1} Anzahl Vergleiche mit Boyer-Moore/KMP 
  \item \ub{7.2} last/failure-Funktion
  \item \ub{7.3} Suche mit Boyer-Moore/KMP
  \item \ub{HS05.2 5} Lösungsansatz pattern-Matching
  \item \ub{HS05.2 6} Boyer-Moore vs. KMP
  \item \ub{HS05.2 7} Boyer-Moore Tabelle
\end{itemize}

\section{Tries}

\begin{itemize}
  \item \ub{8.1} Komprimierten Trie zeichnen
  \item \ubc{8.2} Implementation von Trie-Multimap
\end{itemize}

\section{Dynamic Programming}

\begin{itemize}
  \item \ub{PP 5} LCS
  \item \ub{9.1} Knapsack
  \item \ub{FS08 4} Knapsack
  \item \ubc{FS08 5} Knapsack Implementation
  \item \ub{9.2} LCS
\end{itemize}

\section{Graphen}

\begin{itemize}
  \item \ubc{PP 6} Adjazenzmatrix Implementation
  \item \ubc{PP 7} Floyd-Warshall
  \item \ub{FS08 3} Anzahl Vertex - Kanten min/max
  \item \ub{FS08 3} Adjazenzliste Tiefensuche Laufzeit
  \item \ub{10.1} Adjazenz-Liste vs. -Matrix
  \item \ub{10.2} Anzahl Kanten/Knoten
  \item \ub{10.3} Beweis mit O(log(n bzw. m))
  \item \ub{10.4} Zeitverhalten von Kanten-Listen-Struktur
  \item \ubc{10.5} Graph ADT
  \item \ub{HS05.2 8} Graph von Adjazenzmatrix
  \item \ubc{HS05.2 8.4} Adjuazenzmatrix zu Adjazenzliste
  \item \ubc{HS05.2 10} Gerichtete Graphen: Adjazenzmatrix herleiten
\end{itemize}

\section{Graphen - Traversierung}

\begin{itemize}
  \item \ub{11.1} DFS und BFS
  \item \ubc{11.2} DFS und BFS in Graph-ADT
  \item \ubc{11.3} Web-Crawler
  \item \ub{12.1} Gerichtete Tiefensuche
  \item \ubc{12.2} Implementation: Gerichtete Tiefensuche
  \item \ub{HS05.2 9} BFS-Tabelle
\end{itemize}

\section{Graphen - Djikstra/SPT/MST}

\begin{itemize}
  \item \ub{PP 8} Djikstra
  \item \ub{PP 9} Djikstra Datenstrukturen zeichnen
  \item \ub{PP 10} SPT vs. MST
  \item \ub{13.1} Djikstra
  \item \ubc{13.2} Implementation Djikstra
  \item \ubc{14.1} Pathfinder
  \item \ub{HS05.2 11} Pfad finden
\end{itemize}
\end{document}